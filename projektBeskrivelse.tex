% Opsætter KU Tex
%%%%%%%%%%%%%%%%%%%%%%%%%%%%%%%%%%%%%%%%%%%%%%%%%%%%%%%%%%%%%%%%%%%%%%%%%%%%%%%%
\documentclass{article}
\usepackage[a4paper, hmargin={2.8cm, 2.8cm}, vmargin={2.5cm, 2.5cm}]{geometry}
\usepackage{eso-pic}  % \AddToShipoutPicture
\usepackage{graphicx} % \includegraphics
%%%%%%%%%%%%%%%%%%%%%%%%%%%%%%%%%%%%%%%%%%%%%%%%%%%%%%%%%%%%%%%%%%%%%%%%%%%%%%%%

% Pakker til skrifttyper, tekst osv.
%%%%%%%%%%%%%%%%%%%%%%%%%%%%%%%%%%%%%%%%%%%%%%%%%%%%%%%%%%%%%%%%%%%%%%%%%%%%%%%%
    \usepackage[utf8]{inputenc} % Implementere Unicode
    \usepackage[T1]{fontenc}    % Unicode skrifttype, fx. é skrives som 1 tegn
    \usepackage[danish]{babel}  % Dansk Ordbog
    \usepackage{microtype}      % Forbedre linjeombrydningen
    \usepackage{libertine}      % Skrifttype
    \usepackage[scaled=0.83]{inconsolata} % Skrifttype til kode til kode
%%%%%%%%%%%%%%%%%%%%%%%%%%%%%%%%%%%%%%%%%%%%%%%%%%%%%%%%%%%%%%%%%%%%%%%%%%%%%%%%

% Pakker til matematik og kode.
%%%%%%%%%%%%%%%%%%%%%%%%%%%%%%%%%%%%%%%%%%%%%%%%%%%%%%%%%%%%%%%%%%%%%%%%%%%%%%%%
    \usepackage{mathtools}       % Udvidelse til amsmath pakken
    \usepackage{algpseudocode}   % pseudocode til algoritmer
    \usepackage{algorithm}       % Pakke til algoritmer
    \usepackage{amsthm}          % Pakke til Theroms
%%%%%%%%%%%%%%%%%%%%%%%%%%%%%%%%%%%%%%%%%%%%%%%%%%%%%%%%%%%%%%%%%%%%%%%%%%%%%%%%

% Pakker til layout.
%%%%%%%%%%%%%%%%%%%%%%%%%%%%%%%%%%%%%%%%%%%%%%%%%%%%%%%%%%%%%%%%%%%%%%%%%%%%%%%%
    \usepackage{fancyhdr}        % Gør det muligt at bruge sidehoveder
    \usepackage{graphicx}        % Mulighed for bl.a. \includegraphics
    \usepackage{parskip}         % Første paragraf i afsnit indrykkes ikke
    \usepackage{listings}        % Pakke til at indsætte kode
    \usepackage{enumitem}        % Gør det muligt at tilpasse lister
    \usepackage{titlesec}        % Tilpassing af afstand mellem sektioner
    \usepackage[lastpage,user]{zref} % Side x af y
%%%%%%%%%%%%%%%%%%%%%%%%%%%%%%%%%%%%%%%%%%%%%%%%%%%%%%%%%%%%%%%%%%%%%%%%%%%%%%%%
	\usepackage{url}

% Implementerer en række makroer og de pakker der er importeret
%%%%%%%%%%%%%%%%%%%%%%%%%%%%%%%%%%%%%%%%%%%%%%%%%%%%%%%%%%%%%%%%%%%%%%%%%%%%%%%%
    \pagestyle{fancy}                        % Implementerer sidehoved
    \lhead{Benjamin Rotendahl} % Venstre sidehoved
    \rhead{Hjernespil}      % Højre sidehoved
    \cfoot{\thepage\ of \zpageref{LastPage}} % Side x af y
    \newtheorem*{prp}{Propostion}            % Skaber nyt theorem
    \setlist{nolistsep}              % Formindsker mellemrum mellem listepunkter

    % Definitioner af farver
    %%%%%%%%%%%%%%%%%%%%%%%%%%%%%%%%%%%%%%%%%%%%%%%%%%%%%%%%%%%%%%%%%%%%%%%%%%%%
        \definecolor{KURed1}{RGB}{144,26,30}    % Official KU Red 1
        \definecolor{KURed2}{RGB}{199,36,41}    % Unofficial KU Red
        \definecolor{KUGray1}{RGB}{102,102,102} % Official KU Gray 1
        \definecolor{KUGray2}{RGB}{133,133,133} % Official KU Gray 2
        \definecolor{KUGray3}{RGB}{163,163,163} % Official KU Gray 3
        \definecolor{KUGray4}{RGB}{194,194,194} % Official KU Gray 4
        \definecolor{KUGray5}{RGB}{224,224,224} % Official KU Gray 5
    %%%%%%%%%%%%%%%%%%%%%%%%%%%%%%%%%%%%%%%%%%%%%%%%%%%%%%%%%%%%%%%%%%%%%%%%%%%%

    % Mindsker afstanden mellem sektioner
    %%%%%%%%%%%%%%%%%%%%%%%%%%%%%%%%%%%%%%%%%%%%%%%%%%%%%%%%%%%%%%%%%%%%%%%%%%%%
        \titlespacing\section{0pt}{12pt plus 4pt minus 2pt}
                                  {0pt plus 1pt minus 3pt}
        \titlespacing\subsection{0pt}{12pt plus 4pt minus 2pt}
                                  {0pt plus 1pt minus 3pt}
        \titlespacing\subsubsection{0pt}{12pt plus 4pt minus 2pt}
                                  {0pt plus 1pt minus 3pt}
    %%%%%%%%%%%%%%%%%%%%%%%%%%%%%%%%%%%%%%%%%%%%%%%%%%%%%%%%%%%%%%%%%%%%%%%%%%%%

    % Laver titel
    %%%%%%%%%%%%%%%%%%%%%%%%%%%%%%%%%%%%%%%%%%%%%%%%%%%%%%%%%%%%%%%%%%%%%%%%%%%%
    \title{
      \vspace{13em}
      \Large{Data Training for Librarians} \\
      \Huge{An introduction to programming and data manipulation}
    }

    \author{
        \Large{Arinbjörn Brandsson - arbr@di.ku.dk} \\
        \Large{Benjamin Rotendahl - Benjamin@Rotendahl.dk} \\
    }

    \date{
        \vspace{22em}
        \today \\
    }
    %%%%%%%%%%%%%%%%%%%%%%%%%%%%%%%%%%%%%%%%%%%%%%%%%%%%%%%%%%%%%%%%%%%%%%%%%%%%
%%%%%%%%%%%%%%%%%%%%%%%%%%%%%%%%%%%%%%%%%%%%%%%%%%%%%%%%%%%%%%%%%%%%%%%%%%%%%%%%
%%%%%%%%%%%%%%%%%%%%      Her starter dokumentet    %%%%%%%%%%%%%%%%%%%%%%%%%%%%
\begin{document}


    %% Change `ku-farve` to `nat-farve` to use SCIENCE's old colors or
    %% `natbio-farve` to use SCIENCE's new colors and logo.
    \clearpage

%Disse linjer skaber forside, evt indholdsfortegnelse, og sætter sidetal
%%%%%%%%%%%%%%%%%%%%%%%%%%%%%%%%%%%%%%%%%%%%%%%%%%%%%%%%%%%%%%%%%%%%%%%%%%%%%
    \maketitle              % Forside
    \thispagestyle{empty}   % Fjerner sidetal forside
    \newpage                % Første rigtige side
    \setcounter{page}{1}    % Sætter rigtigt sidetal på første side
%%%%%%%%%%%%%%%%%%%%%%%%%%%%%%%%%%%%%%%%%%%%%%%%%%%%%%%%%%%%%%%%%%%%%%%%%%%%%

\section{Workshop beskrivelse}
    Vi afholder en workshop hvor målet er at give en introduktion til python og
    hvordan det kan bruges til at manipulere med f.eks \emph{CSV} filer.

\section{Program for dagen}
    \begin{description}
        \item[9-10:] Morgenkaffe og hjælp til installation. Før workshoppen
        har vi sendt en guide ud der viser hvordan man installere de nødvendige
        ting så vi kan bruge noget af tiden på at introducere programmet.

        \item[10-10.30:] Fremlæggelse med en introduktion til hvad det vil sige
        at programmere og motivere hvordan programmering kan hjælpe dem.


        \item[10.30-12.00:] Små opgaver der leder workshopsdeltagerne igennem
        de forskellige sprog konstruktioner i python. Der vil blive udleveret et
        opgave sæt hvor forskellige man bliver præsenteret for en
        problemstilling. Vi vil løbende opsamle på opgaverne på tavlen.

        \item[12-12.45:] Frokost

        \item[12.45-13.15:] Introduktion til datahåndtering her lærer man hvordan
        csv filer indlæses og hvilke ting man skal have i mente når man arbejder
        med store datamængder. Der vil være eksempler oppe på projektoren hvor
        man laver simpel analyse på et datasæt.


        \item[13.15-14.30:] En større opgave hvor deltagerne skal
        skrive et program der samler en csv fil og tilføjer forskellige
        opsummeringer af data til rækkerne.

        \item[14.30-16.30:] Deltagerne finder selv på en opgave de ønsker at
        løse, det kan være en opgave fra deres daligdag hvor de kan bruge deres
        nyfunde evner. Alt efter hvor store opgaver de finder på, kan vi evt.
        bruge 30/45 minutter på en introduktion til API'er, så man kan udvide
        sin nuværende data med resultater fra nettet.

    \end{description}

\section{Teknisk information}
    Vi planlægger at bruge Python 3 fortolkeren da vi undgår indkodningsfejl ved
    specialtegn og installationen er nemmere for windows baseret maskiner.
    Som editor bruger vi \emph{atom} der er gratis og open source med udvidelser
    der gør udviklingsprocessen mere gnidningsfrit.


\section{Økonomi}
    Vi forventer at bruge $1.5$ dage på udarbejdelse af slide og opgaver til en
    samlet pris på $7000$kr.

\section{Beskrivelse af os}

% \begin{description}
%     \item[Billeder:] vedhæftet mailen.
%     \item[Navne:] Benjamin Rotendahl \& Arinbjörn Brandsson
%     \item[Affiliation:] Computer Science students at the University of Copenhagen.
%     \item[E-mail:] Benjamin@Rotendahl.dk \&
%     \item[Twitter:] Vi bruger ikke rigtig twitter
%     \item[* Github account if you have one:]
% \end{description}
%
%
%
% * link to blog or other web presence:
% * Title of your session/invited talk
% * 2-3 sentences describing the content of your teaching/invited talk. Specificer her hvis deltagerne skal installere software, programmer, applikationer, osv.




\end{document}
