\documentclass[aspectratio=169,12pt,t]{beamer}
%\usetheme[greyauthor, % Grå tekst forfatter som KU vil have
%         unit=ics, % Ændre til NAT, KU, eller unit=ics (diku)
%         dk, % Sprog
%         %style=simple, % Vandmærke eller billede
%         footstyle=low, % Fjern stor footer
%         wmark, % vandmærke på hver side
%         logoplace=left % Logo til venstre
%         %,sidebar % makes sidebar
%         ]{Frederiksberg}
% nat for Science, ku for generic or unit=ics for DIKU
% Tilføj style=simple for vandmærke
% Default fixed font does not support bold face
\DeclareFixedFont{\ttb}{T1}{txtt}{bx}{n}{12} % for bold
\DeclareFixedFont{\ttm}{T1}{txtt}{m}{n}{12}  % for normal

% Custom colors
\usepackage{color}
\definecolor{deepblue}{rgb}{0,0,0.5}
\definecolor{deepred}{rgb}{0.6,0,0}
\definecolor{deepgreen}{rgb}{0,0.5,0}


\mode<presentation>                                                                              
{                                                                                                
  \usetheme{Copenhagen}                                                                          
  \usecolortheme{default}                                                                        
  \usefonttheme{default}                                                                         
  \setbeamertemplate{navigation symbols}{}                                                       
  \setbeamertemplate{caption}[numbered]                                                          
}             
\usepackage{listings} % Pakke til kode
\usepackage{pslatex}        % pæn skrift
\usepackage[utf8]{inputenc} % Implementerer Unicode
\usepackage{algpseudocode}
\usepackage{algorithm}
\addtobeamertemplate{block begin}{\vspace*{-3pt}}{}
\addtobeamertemplate{block end}{}{\vspace*{-3pt}}

\newcommand\pythonstyle{\lstset{
language=Python,
basicstyle=\ttm,
otherkeywords={self, with},             % Add keywords here
keywordstyle=\ttb\color{deepblue},
emph={MyClass,__init__},          % Custom highlighting
emphstyle=\ttb\color{deepred},    % Custom highlighting style
stringstyle=\color{deepgreen},
frame=tb,                         % Any extra options here
showstringspaces=false            % 
}}


% Python environment
\lstnewenvironment{python}[1][]
{
\pythonstyle
\lstset{#1}
}
{}

\title{Comma-Separated Values (CSV)}
\subtitle{Håndtering, Søgning og Analyse af CSV-Filer}
\author{
        Arinbjörn Brandsson \\
        Benjamin Rotendahl  \\
}

\date[]{\today}


\begin{document}

\frame[plain]{\titlepage}
 \frame{\tableofcontents}

\section{CSV-Filer}

\begin{frame}[fragile]
  \frametitle{Opbygningen af en CSV-Fil}
  \vspace*{-12pt}
  \begin{block}{Første linje beskriver headernavnene, resten er data}
    \begin{lstlisting}
      id,name,description,menus_appeared                                                
      1,Consomme printaniere royal,,8
      2,Chicken gumbo,,111
      3,Tomato aux croutons,,13
    \end{lstlisting}
  \end{block}
  \begin{block}{CSV-Filer kan visualiseres som en tabel}
  \begin{table}
    \begin{tabular}{ | l | l | l | l | }
      \hline
      id&name&description&menus\_appeared\\                                                
      \hline
      1&Consomme printaniere royal&&8\\
      \hline
      2&Chicken gumbo&&111\\
      \hline
      3&Tomato aux croutons&&13\\\hline
    \end{tabular}
  \end{table}
  \end{block}
\end{frame}

\section{CSV-Filer i Programmering}
\subsection{Indlæsning af CSV-Filer}
\begin{frame}[fragile]
  \frametitle{Indlæsning af CSV-Filer}
  \vspace*{-14pt}
  \begin{block}{En måde man kan loade CSV-Filer}
    \begin{python}
# Importer csv biblioteket
import csv
# Aaben filen, og kald den csv_file
with open('filename.csv') as csv_file:
    # Opret en laeser til csv_file, 
    # og kald den reader
    reader = csv.DictReader(csv_file)
  \end{python}
  Et godt tip er at loade filen ind i sin main funktion, og give læseren 
  som argument til andre funktioner. 
  \end{block}
\end{frame}

\begin{frame}[fragile]
  \vspace*{-9pt}
  \begin{block}{Eksempel på en main funktion}
    \begin{python}
# Importer csv biblioteket
import csv

# Kode her...

def main():
    with open('file.csv') as csv_file:
       file_reader = csv.DictReader(csv_file)
       print(findNoget(file_reader))

# Denne if-statement bliver altid kaldt
if __name__ == '__main__':
    main()
    \end{python}
  \end{block}
\end{frame}

\subsection{Bearbejdning af CSV-Data}
\begin{frame}[fragile]
  \frametitle{Bearbejdning af CSV-Data}
  \vspace*{-12pt}
  \begin{block}{Hvordan arbejder man med læseren?}
    \begin{itemize}
      \item Læseren kan læse filens linjer
      \item Læseren kan huske headersne, og har hver linje i en dictionary
      \item Læseren kan ikke ændre i CSV-filen
      \item Så vi skal kunne gennemgå linjerne via læseren
    \end{itemize}
  \end{block}

  \begin{block}{Den oplagte løsning:}
    Vi kan bruge løkker!
    \begin{python}
def findNoget(reader):
    for row in reader:
        print(row['name'])
    \end{python}
  \end{block}
\end{frame}

\begin{frame}[fragile]
  \frametitle{(Simple) Eksempler}
  \vspace*{-14pt}
\begin{block}{Find retter, givet ved name, og returnér deres ID
              (ikke case-sensitive)}
\begin{python}
def findName(reader, name):
    results = []
    for row in reader:
        if row['name'].lower() == name.lower():
            results.append(row['id'])
    return results
\end{python}
\end{block}
\begin{block}{Problem ved denne løsning:}
Tager ikke højde for specialtegn som vi måske vil ignorere, såsom komma 
      (mere om dette senere)
\end{block}
\end{frame}

\begin{frame}[fragile]
  \vspace*{-11pt}
    \begin{block}{Find retter som var sidst i et menukort i 1942}
    \begin{python}
def find1942(reader):
    results = []
    for row in reader:
        if int(row['last_appeared']) == 1942:
            results.append(row['name'])
    return results
    \end{python}
  \end{block}
\begin{block}{Hvad {\tt int()} gør}
  \begin{itemize}
    \item Filens data bliver læst ind som tekststrenge
    \item Uden {\tt int()} ville sammenligningen være {\tt '1942' == 1942}
    \item Hvorimod {\tt int('1942') = 1942}
    \item Det samme kan gøres med {\tt floats}, for eksempel {\tt float('0.8')} 
  \end{itemize}
\end{block}
\end{frame}

\section{Indskrivning af CSV-Filer}
\begin{frame}[fragile]
  \frametitle{Indskrivning af CSV-Filer}
  \begin{block}{Men hvad hvis vi vil lave nye CSV-filer?}
    \begin{itemize}
      \item En læser kan kun læse rækker
      \item Så vi skal bruger en skriver til at skrive i en fil
      \item Virker næsten på samme måde
      \item Man kan enten skrive ind en række ad gangen, eller skrive mange 
            ræker ind på samme tid ({\tt writerow()} og {\tt writerows()})
    \end{itemize}
  \end{block}
\end{frame}
\begin{frame}[fragile]
  \frametitle{Eksempel på at Gemme en Fil}
  \begin{block}{Eksempel: Gem retter som har været i en menu 10 eller flere gange}
    \begin{python}
def save15(reader, fileName):
    with open(fileName, 'w') as save_file:
        headers = reader.fieldnames
        writer = csv.DictWriter(save_file, 
                                fieldnames=headers)
        writer.writeheader()
        for row in reader:
            if int(row['times_appeared']) > 14:
                writer.writerow(row)
    return True 
    \end{python}
  \end{block}
\end{frame}

\section{Problemer og Løsninger med Datasæt}
\begin{frame}
\begin{block}{Mulige problemer man kan løbe ind i:}
  \begin{itemize}
    \item Man har kun brug for ét resultat 
    \item Manglende data i datasættet
    \item Dataen er forkert formatteret % Delimiters?
    \item Dataen indeholder specialtegn
  \end{itemize}
\end{block} 
\end{frame}

\subsection{Ét Resultat}
\begin{frame}[fragile]
\vspace*{-12pt}
  \frametitle{Ét Resultat}
  \begin{block}{Men hvad nu hvis vi kun vil have ét datapunkt?}
    \begin{itemize}
      \item Python har en del kontrolstrukturer - for eksempel til at afbryde 
            løkker tidligt.
      \item Siden vores læser bruger en løkke, kan vi bruge kommandoen 
            {\tt break} til at bryde ud af en løkke øjeblikkeligt
    \end{itemize}
  \end{block}
  \begin{block}{Eksempel - stopper når vi finder id 3}
    \begin{python}
def foo(reader):
    for row in reader:
        if int(row['id']) == 3:
            temp = row
            break
    return temp
    \end{python}
  \end{block}
\end{frame}

\subsection{Manglende Data}
\begin{frame}[fragile]
  \frametitle{Manglende Data}
  \vspace*{-14pt}
    \begin{block}{Hvad gør man hvis der mangler data?}
    \begin{itemize}
      \item Nogle gange mangler datasættet datapunkter
      \item Nogle sammenligninger med manglende datapunkter kan få programmet 
            til at crashe
      \item For at forhindre dette, skal man tage højde for manglende 
            datapunkter (I Python denoteret med den tomme streng, {\tt ''}). 
            Her har man to muligheder:
            \begin{enumerate} 
              \item Inkludere rækken alligevel (nemt)
              \item Ignorere rækken (også nemt)
            \end{enumerate}
      \item I første tilfælde gør man som man har gjort før
      \item I andet tilfælde bruger man kommandoen {\tt continue}, som 
            afbryder den nuværende iteration, og går direkte til den næste 
            iteration.
    \end{itemize}
  \end{block}
\end{frame}

\begin{frame}[fragile]
  \begin{block}{Eksempel fra Før}
    \begin{python}
def findName(reader, name):
    results = []
    for row in reader:
        temp = row['name']
        if temp == '':
            continue
        if temp.lower() == name.lower():
            results.append(row['id'])
    return results
    \end{python}
  \end{block}
  \begin{block}{Bemærk:}
    {\tt continue} kan blive simuleret med {\tt if:... else:...}  
  \end{block}
\end{frame}

\subsection{Forkert-Formatteret Data}
\begin{frame}[fragile]
  \frametitle{Forkert-Formatteret Data}
  \begin{block}{Nogle gange er dataen ikke komma-separeret}
    \begin{itemize}
      \item Nogle gange er dataen separeret med noget andet end et komma
      \item Løsning: Fortæl {\tt csv.DictReader} hvad den nye delimiter er
    \end{itemize}
  \end{block}
  \begin{block}{Eksempel: CSV-fil med punktum i stedet for kommaer}
    \begin{python}
with open('punktummer.csv') as csv_file:
    reader = csv.DictReader(csv_file, delimiter='.')
    # Kode her 
    \end{python}
  \end{block}
\end{frame}

\subsection{Specialkarakterer}
\begin{frame}[fragile]
  \frametitle{Specialkarakterer}
  \begin{block}{Specialkarakterer i ord}
    \begin{itemize}
      \item Nogle gange indeholder dataen specialkarakterer som man ikke er 
            interesseret i
      \item Problematik ved dette: {\tt 'hello,' == 'hello'}
      \item Løsning: Skrub specialkaraktererne fra
      \begin{itemize}
          \item Vi kommer til at bruge sort magi til dette
      \end{itemize}
    \end{itemize}
  \end{block}
\end{frame}

\begin{frame}[fragile]
  \vspace*{-9pt}
  \begin{block}{Eksempel:}
    \begin{python}
import csv
import re
def findName(reader, name):
    results = []
    for row in reader:
        # [^\w] fanger alle tegn som 
        # ikke er bogstaver eller tal
        temp = re.sub(r'[^\w]', '' ,row['name'])
        if temp == '':
            continue
        if  temp.lower() == name.lower():
            results.append(row['id'])
    return results
    \end{python}
  \end{block}
\end{frame}

\end{document}













